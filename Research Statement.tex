\documentclass[11pt]{amsart}

\usepackage{fullpage}
%\usepackage[margin=1in]{geometry}
\usepackage{amsmath,amsfonts,amsthm,amssymb,stmaryrd,paralist,tikz,amsthm}
\usepackage[mathscr]{euscript}
\usetikzlibrary{matrix,arrows,decorations.pathmorphing}

%\usepackage{setspace}
%\doublespacing

\newcommand{\R}{\mathbb{R}}
\newcommand{\Q}{\mathbb{Q}}
\newcommand{\Z}{\mathbb{Z}}
\newcommand{\C}{\mathbb{C}}
\newcommand{\N}{\mathbb{N}}
\newcommand{\D}{\mathbb{D}}
\newcommand{\T}{\mathbb{T}}
\newcommand{\RP}{\mathbb{R}\mathrm{P}}
\newcommand{\CP}{\mathbb{C}\mathrm{P}}
\renewcommand{\H}{\mathbb{H}}
\let\oldS\S
\renewcommand{\S}{\mathbb{S}}
\newcommand{\s}{\mathbb{S}}
\newcommand{\one}{I}
\newcommand{\two}{I\!\!I}
\newcommand{\three}{I\!\!I\!\!I}


\newtheorem{thm}{Theorem}[section]
\newtheorem*{thm*}{Theorem}
\newtheorem{lem}[thm]{Lemma}
\newtheorem*{lem*}{Lemma}
\newtheorem{cor}[thm]{Corollary}
\newtheorem*{cor*}{Corollary}
\newtheorem{prop}[thm]{Proposition}
\newtheorem*{prop*}{Proposition}
\newtheorem{defn}{Definition}
\newtheorem*{defn*}{Definition}
\newtheorem{question}{Question}
\newtheorem*{question*}{Question}
\newtheorem{conj}{Conjecture}
\newtheorem*{conj*}{Conjecture}


\usepackage{color}
\definecolor{verydarkblue}{rgb}{0,0,0.4}
\usepackage{hyperref}
\hypersetup{
pdfauthor={Keaton Quinn},
pdftitle={Asymptotically Poincar\'e surfaces in quasi-Fuchsian manifolds},
colorlinks=true,linkcolor=verydarkblue,
citecolor=verydarkblue,urlcolor=verydarkblue
}

\begin{document}

\title{Research Statement}

\author{Keaton Quinn}


\maketitle

\vspace{-9mm}
\section{Background}

My research lies in differential geometry and global analysis. 
These subjects deal with answering geometric questions using analytic tools, typically from the theory of PDEs. 
These geometric questions are usually with respect to Riemannian manifolds, which are topological spaces modeled on Euclidean space and equipped with a Riemannian metric. 
This metric (which is actually a tensor) induces a notion of distance and angle, letting us ask geometric questions about these spaces. 
Analytic tools are helpful for defining several intuitive concepts. 
For example, the notion of a straight line in this setting is captured by geodesics, and these are locally solutions to a system of nonlinear ODEs. 
The curvature of a Riemannian manifold may be thought of as the obstruction to the manifold being locally isometric to flat Euclidean space and it can be expressed solely in terms of the metric and its derivatives. 
A natural question that frequently arises is whether a given function is the curvature of a Riemannian metric. 
This prescribed curvature question may be answered by constructing solutions to a nonlinear PDE on the manifold. 
It is interesting that the existence of solutions to this PDE depends on the topology of the manifold, and so which curvatures a manifold can support is dictated by its topology. 
My work deals with similar topics, including some prescribed curvature questions, in the setting of hyperbolic manifolds, i.e., Riemannian manifolds whose (sectional) curvature is everywhere equal to $-1$. 

More specifically, my research concerns 3-dimensional hyperbolic geometry.
Thurston's Geometrization Conjecture, which has since been proved, states that every closed 3-manifold may be decomposed into pieces that have one of eight model geometries, one among these being hyperbolic geometry. 
In some sense, the hyperbolic 3-manifolds are the least understood while also being (in ways that can be made precise) the most prevalent of the eight. 
And so, they have received much attention. 
One of the main methods of investigation here is to study the surfaces related---in one way or another---to a hyperbolic 3-manifold. 
These surfaces may be in the context of embedded minimal surfaces (surfaces which locally minimize area), or in foliations of the manifold or its ends, or as surfaces which naturally compactify the 3-manifold. 

Since surfaces are so vital to the study of hyperbolic 3-manifolds, Teichm\"uller theory is frequently employed. 
We denote by $\mathcal{T}(S)$ the Teichm\"uller space of $S$, which is the space of all of its complex structures (modeling $S$ on $\C$), up to an equivalence. 
This space may also be described as the space of equivalence classes of metrics that assign the same angles to tangent vectors (up to homotopy). 
When two metrics assign the same angles we call them conformally equivalent, and $[g]$ will refer to all metrics conformally equivalent $g$ (up to homotopy).


My work so far has focused on quasi-Fuchsian manifolds, which are complete hyperbolic manifolds $M$ diffeomorphic to $S \times (-1,1)$ that contain a non-empty compact geodesically-convex subset. 
Here $S$ is a smooth closed surface of genus greater than 1. 
These manifolds are naturally compactified by two copies of $S$, called the surfaces at infinity, and these copies inherit both complex structures and complex projective structures (modeled on $\CP^1$) induced from $M$. 
Moreover, such a hyperbolic manifold is uniquely determined (up to an equivalence) by the complex structures on these surfaces at infinity. 
And so (see \cite{bers1960}), the space of all such hyperbolic 3-manifolds (up to an equivalence) is in one-to-one correspondence with the product of two copies of $\mathcal{T}(S)$.


For a quasi-Fuchsian manifold, when the complex structures on the surfaces at infinity are complex conjugates, the 3-manifold is called Fuchsian. 
These manifolds contain a unique totally-geodesic minimal surface. 
Almost-Fuchsian manifolds are those manifolds $M$ where the complex structures at infinity are sufficiently close together such that $M$ contains a unique minimal surface (satisfying a principal curvature condition). 
In general, when the complex structures at infinity take any value in $\mathcal{T}(S)$, the manifolds are simply known as quasi-Fuchsian manifolds. 

Foliations have been a successful tool in studying quasi-Fuchsian manifolds.
Indeed, the Renormalized Volume of $M$, a concept from physics, can be defined more geometrically (see for example \cite{krasnov-schlenker2008}) in terms of foliations of the ends of $M$. 
Many have studied the renormalized volume including \cite{schlenker2013},\cite{ciobotaru-moroianu2016}, and \cite{bridgeman-brock-bromberg2019}, where it has been used to obtain bounds on the geometry of the quasi-Fuchsian manifold and the surfaces at infinity. 
In the almost-Fuchsian case, Uhlenbeck showed in \cite{uhlenbeck1983} that parallel copies of the unique minimal surface foliate the entire manifold, and when considered as a path in Teichm\"uller space, this foliation starts and ends at the complex structures on the surfaces at infinity. 
Labourie in \cite{labourie1991} showed the ends of quasi-Fuchsian manifolds admit foliations by constant Gaussian curvature surfaces. 
These constant curvature foliations are where my research began.


\section{Past Research}
In \cite{labourie1991} Labourie proved that hyperbolic ends of 3-manifolds admit foliations by surfaces of constant curvature. 
These surfaces he called $k$-surfaces and for each $k$ in $(-1,0)$ there is a surface of constant Gaussian curvature $k$ belonging to the foliation. 
Each metric with curvature $k$ induces a complex structure and Labourie in \cite{labourie1992} describes the foliation as a path in Teichm\"uller space. 
He shows how the path converges to the conformal class of the hyperbolic metric on the surface at infinity of the end of the manifold (as $k \to 0$).
He asks about the infinitesimal behavior of this path and guesses that it should be related to the projective structure on the surface at infinity. 

His guess is correct. 
In \cite{quinn2020} we make his statement precise and prove it. 
If $I_k$ is the family of first fundamental forms, i.e., the metrics of the $k$-surfaces induced from being submanifolds of $M$, then Labourie showed that the underlying conformal structures $[I_k]$ form a path in Teichm\"uller space that converges to the class of the hyperbolic metric $[h]$ representing the complex structure at infinity. 
These same considerations apply to $I\!I_k$, the second fundamental forms of the $k$-surfaces which also induce points in $\mathcal{T}(S)$. 
We proved the following regarding the infinitesimal behavior of the paths at $[h]$.

\begin{thm}
\label{k-surfaces-thm}
Let $[I_k]$ and $[I\!I_k]$ be the paths of first and second fundamental forms of the $k$-surfaces in Teichm\"uller space. 
Let the complex projective structure at infinity be parametrized by the holomorphic quadratic differential $\phi$. 
Then 
\[
\dot{[I_k]} = -\mathrm{Re}(\phi) \text{ and } \, \dot{[I\!I_k]} = 0.
\]
\end{thm}

The proof uses the construction of Epstein in \cite{epstein1984}, which describes a way to construct surfaces in hyperbolic space given a domain in $\CP^1$ and geometric data on that domain. 
These data take the form of a conformal metric, which can be characterized as a metric conformally equivalent to $h$. 
A variant of this construction gives a surface in $M$ from a conformal metric on the surface at infinity. 
Epstein surfaces have been used by several other others to study the geometry of projective structures and hyperbolic 3-manifolds. 
See, for example, \cite{anderson1998}, \cite{bromberg2004}, and \cite{krasnov-schlenker2008}. 
One of the benefits of these Epstein surfaces is they have a very concrete description in terms of the defining conformal metric and its derivatives. 
Indeed there are explicit formulas for the first and second fundamental forms (seen for example in \cite{dumas2017}) and for the Guassian and mean curvatures (seen for example in \cite{quinn2020}). 

Our main technique in the proof of Theorem \ref{k-surfaces-thm} is to describe $k$-surfaces as Epstein surfaces, at least for $k$ near zero. 
This prescribed curvature problem was done by turning the constant Gaussian curvature $k$ condition into a family, indexed by $k$, of fully nonlinear PDEs in terms of the defining conformal metric. 
This family can be combined to a single PDE in terms of the conformal metric and $k$ and may be rescaled to become a small deformation of the Poincar\'e Theorem (regarding the existence of a hyperbolic metric in a given conformal class, see \cite{tromba1992}). 
And so, an Implicit Function Theorem and elliptic regularity argument shows the PDE has a solution $\sigma_k$ for each $k$ near zero. 
Incidentally, this method also gives another proof of the existence of Labourie's $k$-surface foliation. 
Finally, using the explicit formulas mentioned above, it is then easy to understand $I_k$ and $I\!I_k$ in terms of these solutions. 


Analyzing the asymptotic behavior of $I_k$ and $I\!I_k$ really depended on the fact that we found a family of constants $f(k)$ such that $f(k)\sigma_k$ converges to the hyperbolic metric. 
If one drops the $k$-surfaces setting and just assumes there is a family of conformal metrics $\sigma_\epsilon$ and a family of constants $f(\epsilon)$ such that $f\sigma$ converges to the hyperbolic metric (as $\epsilon \to 0$, in the $C^\infty$ topology), then we get a more general result of a similar kind for the family of Epstein surfaces for the metrics $(\sigma_\epsilon)$. 

\begin{thm}
\label{asym-poincare-thm}
Let $[I_\epsilon]$ and $[I\!I_\epsilon]$ be the the family of first and second fundamental forms representing the Epstein surface in Teichmuller space. 
Then
\[
[I_\epsilon] \to  [h]  \text{ and } \, [I\!I_\epsilon] \to [h]  \text{ as } \epsilon \to 0,
\]
and
\[
\dot{[I_\epsilon]} = -4 f'(0) \mathrm{Re}(\phi) \text{ and } \, \dot{[I\!I_\epsilon]} = 0. 
\]
\end{thm}

Such families we called \emph{Asymptotically Poincar\'e Families} of surfaces since the family corresponding to multiples of the Poincar\'e metric $h$ is a canonical example (where $f(\epsilon) = \epsilon$). 
This Poincar\'e family consists of parallel surfaces, i.e., copies of a surface flowed in its normal direction. 
We showed an asymptotically Poincar\'e family gives a foliation of the end by approximately parallel surfaces.

\begin{thm}
Let $(S_\epsilon)$ be an asymptotically Poincar\'e families of surfaces, then the distance between $S_\epsilon$ flowed for time $t$ in the normal direction and the surface $S_{e^{-2t}\epsilon}$ tends towards zero as $\epsilon$ does.
Moreover, there exists an $\epsilon_0 > 0$ such that for $\epsilon < \epsilon_0$, the surfaces $S_\epsilon$ form a foliation of the end of $M$.
\end{thm}

These results apply to a wide collection of surfaces, a notable example being Labourie's $k$-surfaces. 
They form an asymptotically Poincar\'e family and so Theorem \ref{asym-poincare-thm} has Theorem \ref{k-surfaces-thm} as a corollary. 
Another example is given by the constant mean curvature foliation produced by the work of Mazzeo and Pacard (see \cite{mazzeo-pacard2011}), and so our results characterize the asymptotics of these surfaces as well. 
Our work on the constant mean curvature case also furnishes a new proof of their existence, similar to the $k$-surface case.


\section{Future Research} \label{research plan}


\subsection{Generalizing Asymptotically Poincar\'e Families} \label{generalize families}
A natural extension of my thesis is to consider not just families $(\sigma_\epsilon)$ that converge to $h$ in the asymptotically Poincar\'e sense, but to ask the same questions for families $(\sigma_\epsilon)$ that converge in the same sense to some fixed conformal metric $\sigma_0$. 
We can ask similar questions:
\begin{itemize}
\item Do the first and second fundamental forms still converge to the conformal class at infinity? 
That is, do $[I_\epsilon]$ and $[I\!I_\epsilon]$ converge to $[\sigma_0] = [h]$ as $\epsilon \to 0$?
\end{itemize}
and 
\begin{itemize}
\item What are the tangent vectors to $[I_\epsilon]$ and $[I\!I_\epsilon]$ at $\epsilon =0$. 
How are they related to the Schwarzian derivative of $\sigma_0$ and to the holomorphic quadratic differential $\phi$? 
\end{itemize}
The first of these should follow easily using the same ideas as in Theorem \ref{asym-poincare-thm}. 
Preliminary calculations suggest that yes, $[I(\sigma_\epsilon)]$ and $[\two(\sigma_\epsilon)]$ converge to $[\sigma_0] = [h]$.
Moreover, we expect that it is still the case that $\dot{[\two(\sigma_\epsilon)]} = 0$.
It also appears that $\dot{[I(\sigma_\epsilon)]}$ is indeed related to $\mathrm{Re}(B(\sigma_0))$, where $B(\sigma_0)$.
Here $B(\sigma_0)$, the Schwarzian tensor of Osgood and Stowe \cite{osgood-stowe1992}, is a type of generalization of the Schwarzian derivative of a locally injective holomophic function to the setting of conformally equivalent metrics. 

However, $B(\sigma_0)$ need not be holomorphic, i.e., if $\sigma_0$ does not have constant curvature, so $\mathrm{Re}(B(\sigma_0))$ is not a tangent vector to Teichm\"uller space (see \cite{tromba1992}). The actual tangent vector should be (the real part of) the holomorphic part of $B(\sigma_0)$. 
It would be nice to find a more explicit description of the holomorphic part of a quadratic differential, and to verify these preliminary computations.

\subsection{The Epstein-frame in higher dimensions}
%%%%%%%%%%%
\label{higher-dims}


Epstein's construction in \cite{epstein1984} works in $\H^n$ for $n > 3$ as well, where his formula gives a map into the ball model $\mathbb{B}^n$.
We have computed a formula for the Epstein surface into the hyperboloid model of hyperbolic space as a submanifold of $\R^{n+1}$.
It is given by 
\[
\mathrm{Ep}_\sigma(x) = 
\def\arraystretch{1.5}
\begin{pmatrix}
\frac{1}{2}\left( e^\eta + e^{-\eta} | \nabla \eta |^2 \right)x + e^{-\eta} \nabla \eta \\
\frac{1}{4} \left(e^\eta + e^{-\eta} | \nabla \eta |^2 \right)( | x |^2 - 1) + e^{-\eta}(1 + x \cdot \nabla\eta) \\
\frac{1}{4} \left(e^\eta + e^{-\eta} | \nabla \eta |^2 \right)( | x |^2 + 1) + e^{-\eta}(1 + x \cdot \nabla\eta)
\end{pmatrix}
\]
where the dot product is the Euclidean inner product of $\R^{n+1}$ and the gradient is with respect to this metric.
So far, this has only been verified for $n = 3$ and 4.

It is possible to express this Epstein map as the orbit of a point in $\H^n$ by an $O(n,1)$-frame.
In our case we have $\mathrm{Ep}_\sigma(x) = \widetilde{\mathrm{Ep}}_\sigma(x)p$ for the frame $\widetilde{\mathrm{Ep}}_\sigma: \Omega \to O(n,1)$ given by $\widetilde{\mathrm{Ep}}_\sigma(x) = A(x) B_\sigma(x) C_\sigma(x)$, where 
\begin{align*}
A(x) &= \def\arraystretch{1.5}\begin{pmatrix}
Id_{n-1} & -x & x \\
x^t & 1- \frac{1}{2}|x|^2 & \frac{1}{2} |x|^2 \\
x^t & -\frac{1}{2}|x|^2 & 1 + \frac{1}{2} |x|^2
\end{pmatrix} \\ 
B_\sigma(x) &= 
\def\arraystretch{1.5}
\begin{pmatrix}
Id_{n-1} & \frac{1}{2} \nabla\eta  & \frac{1}{2}\nabla\eta \\
-\frac{1}{2} \nabla\eta^t & 1-\frac{1}{8} |\nabla\eta|^2 & -\frac{1}{8} |\nabla\eta|^2 \\
\frac{1}{2}\nabla\eta^t & \frac{1}{8} |\nabla\eta|^2 & 1 + \frac{1}{8} |\nabla\eta|^2 
\end{pmatrix} \\
C_\sigma(x) &= 
\def\arraystretch{1.5}
\begin{pmatrix}
Id_{n-1} & 0 & 0 \\
0 & \cosh(\eta) & -\sinh(\eta) \\
0 & -\sinh(\eta) & \cosh(\eta) 
\end{pmatrix}
\end{align*}
and the point $p = (0, \ldots, 0 , 3/4,5/4)^t$.

A natural next step would be to use this description of the Epstein map to get closed formulas for the induced metric on the Epstein surface and its curvature, perhaps in terms of the Schwarzian tensor of Osgood and Stowe \cite{osgood-stowe1992}.
We also hope to investigate constant curvature hypersurfaces similarly to our work done in the previous sections using these explicit formulas. 



\subsection{Extension to de Sitter Space}
Via the map $U\H^3 \hookrightarrow T dS^3$ from the unit tangent bundle of hyperbolic space to the tangent bundle of de Sitter space given by $(p,v) \mapsto (v,p)$, a strictly convex surface $S$ with a normal vector field in hyperbolic space has a dual surface $S^*$ in de Sitter space, see \cite{hodgson-rivin1993} or \cite{schlenker2002}. 

This correspondence extends to a duality between hyperbolic ends of 3-manifolds and certain de Sitter space-time 3-manifolds (see \cite{mess2007}).
Therefore, fix a quasi-Fuchsian manifold and an end $E$.
Let $E^*$ be the corresponding de Sitter space-time.
The $k$-surface $S_k$ in $E$ has a dual surface $S_k^*$ in $E^*$ which also has constant Gaussian curvature, as shown by a computation. 
One can show, see \cite{labourie1992}, that the induced metric $I_k^*$ on $S_k^*$ also satisfies $[I_k^*] \to [h]$ in $\mathcal{T}(S)$ as $k \to 0$. 
Hence the paths $[I_k]$ and $[I_k^*]$ meet at $[h]$ when $k = 0$. 

A preliminary calculation shows that $\dot{[I_k^*]} = +\mathrm{Re}(\phi)$, and so if one forms the concatenated path $\gamma : (-1,1) \to \mathcal{T}(S)$ by 
\[
\gamma(t) = 
\begin{cases}
[I_t^*]  & -1 < t \leq 0 \\
[I_{-t}] & 0 \leq t < 1
\end{cases}
\]
then $\gamma'(0) = \mathrm{Re}(\phi)$, so that this path is differentiable. 
Based on this, I conjecture the following, which I intend to prove.

\begin{conj*}
\label{deSitter path}
Let $\gamma: (-1,1) \to \mathcal{T}(S)$ be the concatenated path induced by the $k$-surface foliation and the dual foliation. 
Then $\gamma$ is a smooth curve. 
\end{conj*}


\subsection{Other Equivariant Constructions}

Epstein surfaces are consequences of the fact that $U\H^3$ is isomorphic to the bundle of 1-jets of conformal metrics on $\partial^\infty( \H^3) \simeq \CP^1$, and that this isomorphism is equivariant with respect to the isometry group of $\H^3$. 
Naturally, I wonder
\begin{itemize}

\item Are there other examples of equivariant isomorphisms between a symmetric space or bundles associated to the Frame bundle of the symmetric space and jet spaces of infinitesimal data on the (or a) boundary of the symmetric space? 

\end{itemize}

Moreover, 

\begin{itemize}

\item Even if there is not an isomorphism, are there embeddings of associated bundles on the symmetric space into, say, k-jets of Riemannian metrics on the boundary?

\item Which $k$ is optimal or sharpe?

\item If so, which metrics are in the image of the embedding?

\end{itemize}

One example I will investigate is complex hyperbolic space $\H_{\C}^n$.
Here the Gauss map which sends a unit tangent vector to the ideal endpoint of the geodesic ray in the direction of the vector is still a diffeomorphism. So, we still have a visual metric construction sending the induced metric on a unit tangent sphere to a Riemannian metric on the visual boundary $\partial^\infty (\H_{\C}^n) \simeq S^{2n-1}$. 

\begin{itemize}

\item Are the visual metrics from different points still conformally equivalent as they were in the real hyperbolic setting?

\item What surfaces in $\H_{\C}^n$ can we get out of this construction? How explicitly can we determine the geometry of the surfaces? For example, can we get formulas for the induced metrics? For the curvatures?

\item Can we find an $\mathrm{Isom}(\H_{\C}^n)$ frame-field description of these surfaces similar to that given in \cite{dumas2017} for the real hyperbolic case in dimension 3, or similar to the proposed frame given here in Section \ref{higher-dims}?

\item For $n=2$, in what way does the fact that the boundary at infinity $\partial^\infty \H_{\C}^2$ is the 1-point compactification of the Heisenberg group affect this construction?

\end{itemize}
Once I determine the answers to these questions I can ask how many of the results on foliations of hyperbolic ends transfer to the complex setting. Indeed,
\begin{itemize}

\item Do complex hyperbolic ends admit foliations by constant curvature (hyper)surfaces?

\item Is the asymptotics of such a foliation determined by the geometry of the hyperbolic manifold and the boundary at infinity?

\end{itemize}




\bibliography{references}
\bibliographystyle{amsalpha}

\end{document}