\documentclass[11pt]{amsart}

\usepackage{fullpage}
%\usepackage[margin=1in]{geometry}
\usepackage{amsmath,amsfonts,amsthm,amssymb,stmaryrd,paralist,tikz,amsthm}
\usepackage[mathscr]{euscript}
\usetikzlibrary{matrix,arrows,decorations.pathmorphing}

%\usepackage{setspace}
%\doublespacing

\newcommand{\R}{\mathbb{R}}
\newcommand{\Q}{\mathbb{Q}}
\newcommand{\Z}{\mathbb{Z}}
\newcommand{\C}{\mathbb{C}}
\newcommand{\N}{\mathbb{N}}
\newcommand{\D}{\mathbb{D}}
\newcommand{\T}{\mathbb{T}}
\newcommand{\RP}{\mathbb{R}\mathrm{P}}
\newcommand{\CP}{\mathbb{C}\mathrm{P}}
\renewcommand{\H}{\mathbb{H}}
\let\oldS\S
\renewcommand{\S}{\mathbb{S}}
\newcommand{\s}{\mathbb{S}}
\newcommand{\one}{I}
\newcommand{\two}{I\!\!I}
\newcommand{\three}{I\!\!I\!\!I}


\newtheorem{thm}{Theorem}[section]
\newtheorem*{thm*}{Theorem}
\newtheorem{lem}[thm]{Lemma}
\newtheorem*{lem*}{Lemma}
\newtheorem{cor}[thm]{Corollary}
\newtheorem*{cor*}{Corollary}
\newtheorem{prop}[thm]{Proposition}
\newtheorem*{prop*}{Proposition}
\newtheorem{defn}{Definition}
\newtheorem*{defn*}{Definition}
\newtheorem{question}{Question}
\newtheorem*{question*}{Question}
\newtheorem{conj}{Conjecture}
\newtheorem*{conj*}{Conjecture}


\usepackage{color}
\definecolor{verydarkblue}{rgb}{0,0,0.4}
\usepackage{hyperref}
\hypersetup{
pdfauthor={Keaton Quinn},
pdftitle={Asymptotically Poincar\'e surfaces in quasi-Fuchsian manifolds},
colorlinks=true,linkcolor=verydarkblue,
citecolor=verydarkblue,urlcolor=verydarkblue
}

\begin{document}

\title{Project Description}

\author{Keaton Quinn}


\maketitle

\vspace{-9mm}
\section{Introduction and Background}
\subsection{Introduction}
I propose a postdoctoral research project in the geometry of hyperbolic 3-manifolds and their study via their relationship with the conformal and projective structures on their boundaries. This program will be in collaboration with Ken Bromberg at the University of Utah, and it be a continuation of my thesis research in foliations of ends of hyperbolic 3-manifolds. Bromberg's own work has touched on these topics, where he has used similar ideas to obtain results on the geometry of negatively curved spaces. Working with him and learning from him will lead to the successful resolution of the questions and conjectures posed in this proposal. Moreover, the University of Utah makes for an excellent location for such a collaboration, as it has long been at the forefront of geometric and topological research.

The rest of this section is dedicated to the mathematical setting of the proposal. Section 2 contains a discussion of my past research. This details my work which answers a conjecture of Labourie from \cite{labourie1992} regarding the infinitesimal behavior of a foliation by constant curvature surfaces. This result is then generalized to families we call \emph{Asymptotically Poincar\'e families}, of which Labourie's foliation is an example. Section 3 lays out the proposed research to be done in this program. This includes a natural generalization of my thesis work, an extension of my work in hyperbolic manifolds to de Sitter space, a collaboration with Filippo Mazzoli at the University of Luxembourg to answer a conjecture of Thurston regarding the existence of constant mean curvature foliations of almost-Fuchsian manifolds, and my intention to transfer a common tool in the study of real hyperbolic manifolds to complex hyperbolic manifolds. 

\subsection{Background}
My research lies in 3-dimensional hyperbolic geometry. One of the main methods of investigation here is to study the surfaces related to a hyperbolic manifold. These may take the form of embedded minimal surfaces, foliations of the manifold or its ends by surfaces, or surfaces which naturally compactify the 3-manifold. My work so far has focused on quasi-Fuchsian manifolds, which are complete hyperbolic manifolds $M$ diffeomorphic to $S \times (-1,1)$ that contain a non-empty compact geodesically-convex subset. Here $S$ is a smooth closed surface of genus greater than 1. These manifolds are naturally compactified by two copies of $S$, called the surfaces at infinity, and these copies inherit both conformal structures and complex projective structures induced from $M$. Moreover, such a hyperbolic manifold is (up to an equivalence) uniquely determined by the conformal structures on these surfaces at infinity (this is Ber's Simultaneous Uniformization). And so, the space of all such hyperbolic 3-manifolds (up to an equivalence) is in one-to-one correspondence with two copies of $\mathcal{T}(S)$, the Teichm\"uller space of $S$.

When the conformal structures on the surfaces at infinity are complex conjugates, the 3-manifold is called Fuchsian. These manifolds contain a unique totally-geodesic minimal surface. Almost-Fuchsian manifolds are those manifolds $M$ where the conformal structures at infinity are sufficiently close together such that $M$ contains a unique minimal surface whose principal curvatures are bounded between $-1$ and $1$. In general, when the conformal structures at infinity take any value in $\mathcal{T}(S)$, the manifolds are simply known as quasi-Fuchsian manifolds. 

Foliations have been a successful tool in studying quasi-Fuchsian manifolds. Indeed, the Renormalized Volume of $M$, a concept from physics, can be defined more geometrically (see for example \cite{krasnov-schlenker2008}) in terms of foliations of the ends of $M$. Many have studied the renormalized volume including \cite{schlenker2013},\cite{ciobotaru-moroianu2016}, and Bromberg in \cite{bridgeman-brock-bromberg2019}, where it has been used to obtain bounds on the geometry of the quasi-Fuchsian manifold and the surfaces at infinity. In the almost-Fuchsian case, Uhlenbeck showed in \cite{uhlenbeck1983} that parallel copies of the unique minimal surface foliate the entire manifold, and when considered as a path in Teichm\"uller space, this foliation starts and ends at the conformal structures on the surfaces at infinity. Labourie in \cite{labourie1991} showed the ends of quasi-Fuchsian manifolds admit foliations by constant Gaussian curvature surfaces. These constant curvature foliations are where my research began.


\section{Past Research}
In \cite{labourie1991} Labourie proved that hyperbolic ends of 3-manifolds admit foliations by surfaces of constant curvature. These surfaces he called $k$-surfaces since for each $k$ in $(-1,0)$ there is a surface of constant Gaussian curvature $k$ belonging to the foliation. In \cite{labourie1992} Labourie describes the foliation as a path in Teichm\"uller space and shows how the path converges to the conformal class of the hyperbolic metric on the surface at infinity of the end of the manifold. He asks about the tangent vector to this path and guesses that it should be related to the projective structure on the surface at infinity. 

His guess is correct. In \cite{quinn2019} we make his statement precise and prove it. If $I_k$ is the family of first fundamental forms of the $k$-surfaces, then the underlying conformal structures $[I_k]$ form a path in Teichm\"uller space that converges to the class of the hyperbolic metric $[h]$ representing the complex structure at infinity. These same considerations apply to $I\!I_k$, the second fundamental forms of the $k$-surfaces, which are negative definite and so also define points in $\mathcal{T}(S)$. We proved the following regarding the infinitesimal behavior of the paths at $[h]$.

\begin{thm}
\label{k-surfaces-thm}
Let $[I_k]$ and $[I\!I_k]$ be the paths of first and second fundamental forms of the $k$-surfaces in Teichm\"uller space. Let the complex projective structure at infinity be parametrized by the holomorphic quadratic differential $\phi$. Then 
\[
\dot{[I_k]} = -\mathrm{Re}(\phi) \text{ and } \, \dot{[I\!I_k]} = 0.
\]
\end{thm}

The proof uses the construction of Epstein in \cite{epstein1984}, which describes a way to construct surfaces in hyperbolic space given a domain in $\CP^1$ and a conformal metric on that domain. A variant of this construction gives a surface in $M$ from a conformal metric on the surface at infinity. One of the benefits of these surfaces is they have a very concrete description in terms of the defining conformal metric and its derivatives. Indeed there are explicit formulas for the first and second fundamental forms (seen for example in \cite{dumas2017}) and for the Guassian and mean curvatures (seen for example in \cite{quinn2019}). 

Our main technique in the proof of Theorem \ref{k-surfaces-thm} is to describe $k$-surfaces as Epstein surfaces, at least for $k$ near zero. This was done by turning the constant Gaussian curvature $k$ condition into a family of PDEs in terms of the defining conformal metric and $k$. An Implicit Function Theorem and elliptic regularity argument shows the PDE has a solution $\sigma_k$ for $k$ near zero. It is then easy to understand $I_k$ and $I\!I_k$ in terms of these solutions.

The proof really depended on the fact that we found a family of constants $f(k)$ such that $f(k)\sigma_k$ converges to the hyperbolic metric. If one drops the $k$-surfaces setting and just assumes there is a family of conformal metrics $\sigma_\epsilon$ and a family of constants $f(\epsilon)$ such that $f\sigma$ converges to the hyperbolic metric as $\epsilon \to 0$ in the $C^\infty$ topology, then we get a more general result of a similar kind. 

\begin{thm}
\label{asym-poincare-thm}
Let $[I_\epsilon]$ and $[I\!I_\epsilon]$ be the the family of first and second fundamental forms representing the Epstein surface in Teichmuller space. Then
\[
[I_\epsilon] \to  [h]  \text{ and } \, [I\!I_\epsilon] \to [h]  \text{ as } \epsilon \to 0,
\]
and
\[
\dot{[I_\epsilon]} = -4 f'(0) \mathrm{Re}(\phi) \text{ and } \, \dot{[I\!I_\epsilon]} = 0. 
\]
\end{thm}

Such families we called \emph{Asymptotically Poincar\'e Families of Surfaces} since the family corresponding to multiples of the Poincar\'e metric is a canonical example (where $f(\epsilon) = \epsilon$). We showed any such family gives a foliation of the end by approximately parallel surfaces.
\begin{thm}
Let $(S_\epsilon)$ be an asymptotically Poincar\'e families of surfaces, then the distance between $S_\epsilon$ flowed for time $t$ in the normal direction and the surface $S_{e^{-2t}\epsilon}$ tends towards zero as $\epsilon$ does.
Moreover, there exists an $\epsilon_0 > 0$ such that for $\epsilon < \epsilon_0$, the surfaces $S_\epsilon$ form a foliation of the end of $M$.
\end{thm}


Labourie's $k$-surfaces form an asymptotically Poincar\'e family and so Theorem \ref{asym-poincare-thm} has Theorem \ref{k-surfaces-thm} as a corollary. Another example is given by the constant mean curvature foliation produced by the work of Mazzeo and Pacard (see \cite{mazzeo-pacard2011}), and so our results characterize the asymptotics of these surfaces as well.
 


\section{Research Plan} \label{research plan}


\subsection{Generalizing Asymptotically Poincar\'e Families} \label{generalize families}
A natural extension of my thesis is to consider not just families $(\sigma_\epsilon)$ that converge to $h$ in the asymptotically Poincar\'e sense, but to ask the same questions for families $(\sigma_\epsilon)$ that converge in the same sense to some fixed conformal metric $\sigma_0$. 
We can ask similar questions:
\begin{itemize}
\item Do $[I_\epsilon]$ and $[I\!I_\epsilon]$ converge to $[\sigma_0] = [h]$ as $\epsilon \to 0$?
\end{itemize}
and 
\begin{itemize}
\item What are the tangent vectors to $[I_\epsilon]$ and $[I\!I_\epsilon]$ at $\epsilon =0$. How are they related to the Schwarzian derivative of $\sigma_0$ and to the holomorphic quadratic differential $\phi$? 
\end{itemize}
The first of these should follow easily using the same ideas as in Theorem \ref{asym-poincare-thm}. The second will need tools from the deformation theory of geometric structures to resolve. These questions were partially inspired by conversations with Ken Bromberg, so this would be a natural starting point for a collaboration to answer them. 

\subsection{Extension to de Sitter Space}
Via the map $U\H^3 \hookrightarrow T dS^3$ from the unit tangent bundle of hyperbolic space to the tangent bundle of de Sitter space given by $(p,v) \mapsto (v,p)$, a surface $S$ with a normal vector field in hyperbolic space has a dual surface $S^*$ in de Sitter space. And so, each $k$-surface in hyperbolic space has a dual surface in de Sitter space. It turns out the dual surface to a $k$-surface also has constant Gaussian curvature and this family of dual $k$-surfaces foliates an end of de Sitter space. Moreover, this family may similarly be considered as a path in Teichm\"uller space. One can show that the dual family also converges to $[h]$ and so these two paths meet at $[h]$ when $k = 0$. One may form the concatenated path $\gamma :(-1,1) \to \mathcal{T}$ with $\gamma(0) = [h]$. Then $\gamma$ is continuous. A preliminary calculation using my results appears to show that this path is differentiable at $[h]$. Based on this, I conjecture the following, which I intend to prove.

\begin{conj}
\label{deSitter path}
Let $\gamma: (-1,1) \to \mathcal{T}(S)$ be the concatenated path induced by the $k$-surface foliation and the dual foliation. Then $\gamma$ is a smooth curve. 
\end{conj}



\subsection{Constant Mean Curvature Foliations}
One may use $k$-surfaces to parametrize the set of hyperbolic ends (See \cite{labourie1992}). Filippo Mazzoli is using my work (in \cite{mazzoli2019}) to show that as $k \to 0$, this parametrization limits to the Schwarzian parametrization of hyperbolic ends. Combining my work and his, he has shown that a foliation by $k$-surfaces can be seen as an integral curve of a time-dependent Hamiltonian vector field on $T^*\mathcal{T}(S)$ (with respect to the canonical symplectic form on the cotangent bundle). Similar arguments give the same results for constant mean curvature foliations. 

The constant mean curvature case is especially interesting. Foliations by constant mean curvature surfaces have been widely used to study the geometry of 3-manifolds. For example, in certain spacetimes, they may be used to define time functions. A natural question, attributed to Thurston in \cite{huang-wang2013}, is
whether quasi-Fuchsian manifolds admit foliations by constant mean curvature surfaces. In \cite{huang-wang2013} Huang and Wang constructed a quasi-Fuchsian manifold which does not admit such a foliation. However, this manifold contains two minimal surfaces and so the question may still be asked for the class of almost-Fuchsian manifolds---those which contain a unique minimal surface with principal curvatures in $(-1,1)$.
\begin{conj}[Thurston]
\label{almost Fuchsian cmc}
An almost-Fuchsian manifold admits a foliation by constant mean curvature surfaces.
\end{conj}
A proof of this conjecture was announced in 2008 in the thesis of Biao Wang \cite{wang2008}, but has not yet appeared in a peer-reviewed journal. Andersson, Moncrief, and Tromba have proved certain spacetimes admit constant mean curvature foliations by showing that the foliation may be interpreted as an integral curve of a time dependent vector field and then showing this vector field is complete (see \cite{andersson-moncrief-tromba1997}). Mazzoli and I expect to use these same ideas in the hyperbolic setting to prove this conjecture. As a preliminary step we hope to answer: 
\begin{itemize}
\item Can a new proof of the existence of $k$-surface foliations of hyperbolic ends be given using these techniques?
\end{itemize}

\subsection{Complex Hyperbolic Space}
Finally, I am interested in how many of these questions can be answered in the complex hyperbolic setting. For example:
\begin{itemize}
\item Do complex hyperbolic ends admit foliations by constant curvature (hyper)surfaces?
\item To this end, do any of the same tools transfer to the complex setting? 
\end{itemize}
For my work, Epstein surfaces have been quite integral. So I ask:
\begin{itemize}
\item Is there an Epstein-like construction for complex hyperbolic space, i.e., an analogous way of taking geometric data on the ideal boundary $\partial^\infty \H_{\C}^n$ and constructing a (hyper)surface in $\H_{\C}^n$ such that this construction is equivariant with respect to the isometry group? 
\item For $n=2$, in what way does the fact that the boundary at infinity $\partial^\infty \H_{\C}^2$ is the 1-point compactification of the Heisenberg group affect this construction?
\end{itemize}
I intend to find a $\mathrm{Isom}(\H_{\C}^2)$ frame-field description of these complex Epstein surfaces, similar to that given in \cite{dumas2017} for the real hyperbolic case, and use this to answer these questions.





\section{Career Development}
My current career plans include obtaining a tenure track professorship researching mathematics. To this end, this fellowship would help me secure such a position by allowing me to produce 4 papers of a similar quality to \cite{quinn2019}. This will ideally allow me to bypass the need for a second postdoctoral position. Furthermore, several of the goals mentioned in Section \ref{research plan} will require me to learn new techniques. Tools from pseudo-Riemannian geometry will be needed to investigate Conjecture \ref{deSitter path}. Harmonic maps are needed for the parametrization of hyperbolic ends by $k$-surfaces. Topics from symplectic geometry will be helpful for Conjecture \ref{almost Fuchsian cmc}, as these time-dependent vector fields are Hamiltonian. Ideas from complex geometry beyond Riemann surfaces are needed to investigate Epstein-like constructions in $\H_{\C}^2$. Awarding this proposal would afford me a research year to learn these topics both through self-study and by consulting experts at the University of Utah. This will lead to more productive future years where the preceding questions and conjectures may be answered quickly. 








\section{Sponsoring Scientist and Host Institution}
Bromberg and I have overlapping research interests. These include the comparison of intrinsic versus extrinsic geometry of surfaces in 3-manifolds, the interplay between Teichm\"uller theory and hyperbolic 3-manifolds, and more generally, the deformation theory of geometric structures on surfaces and 3-manifolds. Indeed, Bromberg in \cite{bromberg2004} used deformations of both hyperbolic structures and complex projective structures to study a moduli space of hyperbolic cone metrics on a fixed topological manifold. More recently, in \cite{brock-bromberg2016}, he helped make connections between the geometry of certain hyperbolic 3-manifolds and the Weil-Petersson geometry of the boundary surfaces at infinity. And, in \cite{bridgeman-brock-bromberg2019}, he helped prove that the renormalized volume of quasi-Fuchsian manifolds is non-negative using gradient flows on Teichm\"uller space. Epstein surfaces serve as a common tool in Bromberg's work. Hence, Bromberg is a good fit for me as a postdoctoral sponsor. Moreover, he has been supportive of my work in the past. In particular, the project described in Section \ref{generalize families} was partially inspired by a conversation with him regarding my research.

In addition to this, the University of Utah has a history of producing excellent research in geometry, topology, and dynamics. Beyond Bromberg, Utah employs many excellent mathematicians in these fields, including, but not limited to, Mladen Bestvina, Jon Chaika, Priyam Patel, and Domingo Toledo. Hence, the University of Utah will provide an ideal environment to carry out the research proposed here. 


\section{Broader Impact}

Recently, hyperbolic geometry has been used to study data structures, machine learning, and neural networks. Embedding data in some Euclidean space has been a fruitful idea since it allows the tools of vector calculus to be applied. However, for some data sets, particularly those where hierarchy is important, these tools can become inefficient. Negatively curved spaces, as opposed to flat spaces, may allow some data sets, represented as graphs, to be embedded isometrically, or at least with bounded error. Standard neural network tools are being adapted to the hyperbolic setting where they out perform their Euclidean counterparts (see \cite{ganea2018} or \cite{desa2018} for examples). Deep learning techniques have been widely successful tools for solving complicated problems. For example, deep learning algorithms have been used to diagnose medical issues, where they may be as accurate as health care professional (see \cite{xiaoxuan-faes2019} for a review). And so, optimizations of these tools using hyperbolic geometry can have a large effect on society. Therefore, by expanding the range of tools available to understand negatively curved geometric objects, my research has the potential for broader impacts by supporting these emerging areas in which such geometric techniques may be applied.



More personally, I intend to create events to increase the participation of underrepresented minorities in math. Indeed, members of the LGBTQ community are estimated to be 17-21\% less represented in STEM than would be expected. A discussion can be found in  \cite{freeman2018} where the proposed main reason for this is a lack of visibility. In an attempt to create a sense of community for LGBTQ graduate students I started the Out in Math seminar at UIC. We are in the process of becoming the first student chapter of Spectra: the Associations for LGBTQ+ mathematicians. In June 2019,  Autumn Kent, Harrison Bray, Diana Davis, Talia Fern\'os, and Thomas Koberda organized the LG\&TBQ conference at the University of Michigan. This conference was aimed at sexual and gender minorities interested in geometry and topology and, in addition to math talks, featured discussions related to discrimination faced by these minorities in the math community and ways to make the math community more welcoming. After the success of this conference, there has been interest in more events aiming to increase LGBTQ representation. In addition to the Out in Math seminar, I helped organize the 2018 Graduate Student Geometry and Topology Conference at UIC. This conference had over 100 participants and gave graduate students a chance to connect with others in their research areas. I hope to use the experience I gained from organizing these seminars and conferences to help create more events similar to LG\&TBQ.  


\bibliography{references}
\bibliographystyle{amsalpha}
















\end{document}